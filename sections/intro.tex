\section{Introducción}

\frame{
\frametitle{¡Contribuye!}
  El código fuente de éstas diapositivas está disponible en:
\begin{center}
  \huge \href{http://github.com/pbaeyens/beamer}{github.com/pbaeyens/beamer}
\end{center}
  Erratas, correcciones y aportaciones son bienvenidas.
}

\frame{
  \frametitle{¿Qué es \beamer?}
  \beamer es una clase de documento de $\LaTeX$ que genera diapositivas
  o transparencias con animaciones, gráficos, tablas...

  Puede compilarse con casi cualquier compilador de $\LaTeX$ y personalizarse al detalle.

  \pause

    \begin{block}{\texttt{pandoc}}
      \texttt{pandoc} no sirve \frownie{}. Podemos generar presentaciones con \beamer
      pero tiene \href{http://johnmacfarlane.net/pandoc/demo/example9/producing-slide-shows-with-pandoc}{una sintaxis propia}.
    \end{block}
}

\frame{
  \frametitle{Instalación}

  Para usar \beamer se necesitan 3 paquetes:

  \begin{itemize}
    \item \beamer
    \item \texttt{pgf}
    \item \texttt{xcolor}
  \end{itemize}

    \pause

  \begin{block}{Debian/Ubuntu y derivados}
    En Debian y derivados podemos instalar \texttt{latex-beamer}. También podemos
    utilizar \href{http://www.ctan.org/pkg/texliveonfly}{\texttt{texliveonfly}}.
  \end{block}
}

\frame{
  \frametitle{Primeros pasos}

  Para empezar a usar \beamer, indicamos la clase del documento:

    \begin{center}
      \Large \texttt
      {\color{black}\textbackslash}{\color{keywords}documentclass}{\color{black}\{beamer\}}
    \end{center}
}

\frame{
  \frametitle{Diapositivas}
  Creamos las diapositivas utilizando el entorno o la función \texttt{frame}:

  \ejemplo{ejemplo1.tex}

  \pause
  \begin{block}{Tipos de diapositivas}
    \begin{itemize}[<+->]
      \item \texttt{shrink}:  Reduce el tamaño para introducir más contenido.
      \item \texttt{plain}:   Diapositiva simple, útil para imágenes.
      \item \texttt{fragile}: Necesario para mostrar \texttt{verbatim}.
      \item \texttt{allowframebreaks}: Divide el contenido en diapositivas.
    \end{itemize}
  \end{block}
}
