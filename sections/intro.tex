\section{Introducción}

\subsection{Instalación}

\begin{frame}{¡Contribuye!}
  El código fuente de éstas diapositivas está disponible en:
\begin{center}
  \huge \href{http://github.com/pbaeyens/beamer}{github.com/pbaeyens/beamer}
\end{center}
  Erratas, correcciones y aportaciones son bienvenidas.
\end{frame}

\begin{frame}{¿Qué es $\LaTeX$}
  $\LaTeX$ es un lenguaje de marcado que permite crear documentos. Los elementos
  básicos dentro del código de un documento son:
  \espacio
  \begin{overlayarea}{\textwidth}{5cm}
      \only<1>{\begin{block}{Comandos}
        El comando \texttt{com} se llama incluyendo \comando{com}.
        \espacio
        \begin{description}
          \item[Argumentos] \comando{com}\texttt{\{arg1\}\{arg2\}}
          \item[Opciones]   \comando{com}\texttt{[op]\{arg\}}
        \end{description}
        Algunos comandos sólo son válidos en ciertos entornos.
      \end{block}}

      \only<2>{\begin{block}{Entornos}
        Un entorno es una sección del documento que permite el uso de ciertos comandos:
        \begin{center}
          \ejemplo{entorno.tex}
        \end{center}
      \end{block}}
  \end{overlayarea}
\end{frame}

\begin{frame}{Estructura básica de un documento $\LaTeX$}
  La estructura básica de un documento $\LaTeX$ consta de 2 partes:
  \espacio
  \begin{description}
    \item[Clase de documento] Se indica con \comando{documentclass}\texttt{\{clase\}}.
    \item[Paquetes y opciones] Se incluyen con \comando{usepackage}\texttt{\{paquete\}}.
    \item[Documento] Se escribe dentro del entorno \texttt{document}.
  \end{description}
\end{frame}

\begin{frame}{¿Qué es \beamer?}
  \href{https://www.ctan.org/pkg/beamer}{\beamer} es una clase de documento de $\LaTeX$
  que genera diapositivas o transparencias.

  Puede compilarse con casi cualquier compilador de $\LaTeX$ y personalizarse al detalle.

  \pause

    \begin{block}{\texttt{pandoc}}
      \texttt{pandoc} no sirve \frownie{}. Podemos generar presentaciones
       con \beamer pero tiene
    \href{http://johnmacfarlane.net/pandoc/demo/example9/producing-slide-shows-with-pandoc}{%
    una sintaxis propia}.
    \end{block}
\end{frame}

\begin{frame}{Instalación}
  Para usar \beamer se necesitan 3 paquetes:

  \begin{itemize}
    \item \beamer
    \item \texttt{pgf}
    \item \texttt{xcolor}
  \end{itemize}

    \pause

  \begin{block}{Debian/Ubuntu y derivados}
    En Debian y derivados podemos instalar \texttt{latex-beamer}. También podemos
    utilizar \href{http://www.ctan.org/pkg/texliveonfly}{\texttt{texliveonfly}}.
  \end{block}
\end{frame}

\subsection{Primeros pasos}

\begin{frame}{Primeros pasos}

  Para empezar a usar \beamer, indicamos la clase del documento:

    \begin{center}
      \Large \texttt
      {\color{black}\textbackslash}{\color{keywords}documentclass}{\color{black}\{beamer\}}
    \end{center}

  \pause
  \begin{exampleblock}{En español}
    Si vamos a escribir en español lo indicamos con:
    \ejemplo{idioma.tex}
  \end{exampleblock}
\end{frame}

\begin{frame}{Diapositivas}
  Los documentos de \beamer se dividen en diapositivas. Las creamos utilizando
  el entorno \texttt{frame}:

  \ejemplo{ejemplo1.tex}

  \pause
  \begin{block}{Tipos de diapositivas}
    \begin{itemize}
      \item \texttt{shrink}:  Reduce el tamaño para introducir más contenido.
      \item \texttt{plain}:   Diapositiva simple, útil para imágenes.
      \item \texttt{fragile}: Necesario para mostrar \texttt{verbatim}.
      \item \texttt{allowframebreaks}: Divide el contenido en diapositivas.
    \end{itemize}
  \end{block}
\end{frame}
