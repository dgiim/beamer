\section{Aspecto}

\subsection{Temas}
% TBD
% General, Interno, Externo, Color

\subsection{Bloques}

\begin{frame}
  \beamer permite crear bloques para estructurar la información:
  \espacio
  \begin{columns}
   \column{.5\textwidth}
      \pauses
      \begin{block}{Bloques normales}
        Se crean con el entorno \\ \texttt{block}.
      \end{block}

      \pause
      \begin{alertblock}{Bloques alerta}
        Se crean con el entorno \texttt{alertblock}.
      \end{alertblock}
   \column{.5\textwidth}
    \pause
    \begin{exampleblock}{Bloques ejemplo}
      Se crean con el entorno \texttt{exampleblock}.
    \end{exampleblock}

    \pause
    \begin{theorem}
      No existen números mayores que 2.
    \end{theorem}
  \end{columns}

  \pause
  \espacio
  Los bloques matemáticos (teoremas, demostraciones...) tienen el mismo aspecto
  que los bloques normales.
\end{frame}

\subsection{Velos} %TBD
\subsection{Columnas} %TBD

\subsection{Formato}

\frame{
\frametitle{Tamaño y color}
\begin{columns}
  \column{.5\textwidth}
    Podemos cambiar el tamaño de letra utilizando los comandos habituales en $\LaTeX$.
    También podemos cambiar el color, utilizando el paquete \texttt{xcolor}.
    \espacio
    Los colores básicos son: {\color{white} blanco}, {\color{black} negro},
    {\color{red} rojo}, {\color{green} verde}, {\color{blue} azul},
    {\color{cyan} cian}, {\color{magenta} magenta} y {\color{yellow} amarillo},
    aunque se pueden \href{http://en.wikibooks.org/wiki/LaTeX/Colors}{ampliar y combinar}.

  \column{.5\textwidth}
    \begin{itemize}
      \item \texttt{\tiny \textbackslash tiny}
      \item \texttt{\scriptsize \textbackslash scriptsize}
      \item \texttt{\footnotesize \textbackslash footnotesize}
      \item \texttt{\small \textbackslash small}
      \item \texttt{\normalsize \textbackslash normalsize}
      \item \texttt{\large \textbackslash large}
      \item \texttt{\Large \textbackslash Large}
      \item \texttt{\LARGE \textbackslash LARGE}
      \item \texttt{\huge \textbackslash huge}
    \end{itemize}
\end{columns}
}

% Matemáticas
