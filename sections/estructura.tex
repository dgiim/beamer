\section{Estructura}

\subsection{Bloques}

\begin{frame}
  \beamer permite crear bloques para estructurar la información:
  \espacio
  \begin{columns}
   \column{.5\textwidth}
      \pause
      \begin{block}{Bloques normales}
        Se crean con el entorno \\ \texttt{block}.
      \end{block}

      \pause
      \begin{alertblock}{Bloques alerta}
        Se crean con el entorno \texttt{alertblock}.
      \end{alertblock}

   \column{.5\textwidth}
    \pause
    \begin{exampleblock}{Bloques ejemplo}
      Se crean con el entorno \texttt{exampleblock}.
    \end{exampleblock}

    \pause
    \begin{theorem}
      No existen números mayores que 2.
    \end{theorem}

  \end{columns}

  \pause
  \espacio
  Los bloques matemáticos (teoremas, demostraciones \dots) tienen el mismo aspecto
  que los bloques normales.
\end{frame}

\subsection{Overlays}

\begin{frame}[fragile]{\texttt{\textbackslash pause}}
  El comando \texttt{pause} permite insertar pausas para mostrar los elementos en
  una diapositiva:

  \espacio
  \begin{columns}
    \column{.5\textwidth}
      \ejemplo{pause.tex}
    \column{.4\textwidth}
      \muestra{pause.tex}
  \end{columns}
\end{frame}

\begin{frame}[fragile]{\textit{Overlays}}{Entornos de enumeración}
  Los \textit{overlays} permiten mostrar elementos selectivamente.
  Pueden utilizarse en casi cualquier elemento de \beamer.

  \espacio

  \begin{exampleblock}{\textit{Overlays} en \texttt{itemize}}

  \begin{columns}
    \column{.3\textwidth}
      \muestra{overlay.tex}
    \column{.5\textwidth}
      \ejemplo{overlay.tex}
  \end{columns}
  \espacio
  Si indicamos \texttt{[<+->]} los elementos aparecerán secuencialmente.
  \end{exampleblock}
\end{frame}

\begin{frame}{\textit{Overlays}}{Formato}
  \setbeamercovered{invisible}
  La sintaxis de los \textit{overlays} permite indicar conjuntos de diapositivas
  o intervalos. Podemos utilizarlos en muchos elementos y con órdenes que nos permiten:
    \espacio

    \begin{itemize}
      \item \texttt{\textbackslash{\color{keywords}textbf}\alert<1>{<2>}\{2\}}
      produce  negrita en la diapositiva \textbf<2>{2}.

      \item
      \texttt{\textbackslash{\color{keywords}alert}\alert<3>{<4->}%
      \{\$\textbackslash sum n\$\}}
      produce \alert<4->{$\sum n$}.

      \item Con \texttt{\textbackslash{\color{keywords}begin}\{block\}\alert<5>{<6->}} el
      bloque no aparece hasta la cuarta diapositiva.
    \end{itemize}

    \begin{block}<6->{Bloque}
      Texto.
    \end{block}
\end{frame}

\begin{frame}[fragile]{\textit{Overlays}}{Otros comandos}
  Utilizando la misma sintaxis, tenemos otros comandos:
  \ejemplo{overlaycom.tex}
  \espacio
  \muestra{overlaycom.tex}
\end{frame}

\begin{frame}{Overlays}{Ajustando overlays}
  Para ajustar el comportamiento de los \textit{overlays} con la orden \texttt{setbeamercovered}:
  \espacio

  \begin{block}{}
    \begin{description}
      \item[transparent] Reduce la opacidad de los elementos cubiertos.
      \item[invisible]   Los elementos cubiertos no se muestran.
      \item[dynamic]     Los elementos más \textit{lejanos} se ven menos.
    \end{description}
  \end{block}
\end{frame}

\begin{frame}{\textit{Overlays} en \texttt{tikz}}{Copiado vilmente de %
\href{http://tex.stackexchange.com/questions/55806}{tex.stackexchange.com}}
  \begin{center}
    \begin{tikzpicture}[mindmap, concept color=gray!50, font=\sf, text=white]
      \tikzstyle{level 1 concept}+=[font=\sf, sibling angle=90,level distance = 30mm]
      \node[concept,scale=0.7] {También podemos}
      [clockwise from=135]
      child[concept color=orange, visible on=<2->]{ node[concept,scale=0.7]{usar los} }
      child[concept color=orange, visible on=<3->]{ node[concept,scale=0.7]{\textit{overlays}}}
      child[concept color=orange, visible on=<4->]{ node[concept,scale=0.7]{en} }
      child[concept color=orange, visible on=<5->]{ node[concept,scale=0.7]{\texttt{tikz}} };
    \end{tikzpicture}
  \end{center}
\end{frame}

\subsection{Columnas}

\begin{frame}{Columnas}
  El entorno \texttt{columns} nos permite organizar la disposición de los
  elementos en una diapositiva o entorno en un número arbitrario de columnas.
  \espacio
  \begin{overlayarea}{\textwidth}{5cm}
    \only<1>{\ejemplo{cols.tex}}
    \only<2>{\espacio \espacio \muestra{cols.tex}}
  \end{overlayarea}
\end{frame}
