% Autor: Pablo Baeyens (@pbaeyens)
% Email: pbaeyens31+github@gmail.com
% Licencia: CC BY-SA 3.0

%% Paquetes y configuración %

% Beamer
\documentclass{beamer}

% Idioma
\usepackage[spanish]{babel} % Traducciones
\usepackage[utf8]{inputenc} % Uso de caracteres UTF-8

% Matemáticas
\usepackage{amsfonts}
\usepackage{amsmath}
\usepackage{amssymb}

% Colores
\definecolor{backg}{HTML}{F2F2F2}    % Fondo
\definecolor{comments}{HTML}{BDBDBD} % Comentarios
\definecolor{keywords}{HTML}{8A0808} % Palabras clave
\definecolor{strings}{HTML}{FA5858}  % Strings
\definecolor{links}{HTML}{DF0101}    % Enlaces
\definecolor{bars}{HTML}{B40404}     % Barras (gráfico)

% Código
\usepackage{listings}
\lstset{
%frame=shadowbox,
language=[LaTeX]TeX,
basicstyle=\footnotesize,
morekeywords={frametitle,framesubtitle, href},
breaklines=true,
backgroundcolor=\color{backg},
keywordstyle=\color{keywords},
commentstyle=\color{comments},
stringstyle=\color{strings},
tabsize=2,
% Incluir símbolos no ASCII (tex.stackexchange.com/questions/24528)
extendedchars=true,
literate={á}{{\'a}}1 {é}{{\'e}}1 {í}{{\'i}}1 {ó}{{\'o}}1
         {ú}{{\'u}}1 {ñ}{{\~n}}1 {¡}{{\textexclamdown}}1
         {¿}{{?`}}1}
}


% Gráficos
\usepackage{pgfplots}
\pgfplotsset{width=7cm,compat=1.8} % Opciones para gráficos

% Emoticonos
\usepackage{wasysym}

%% Comandos %%
% Comandos para incluir ejemplos como listings y mostrarlos
\newcommand{\ejemplo}[1]{\lstinputlisting{./Ejemplos/#1}}
\newcommand{\muestra}[1]{\input{./Ejemplos/#1}}
\newcommand{\espacio}{\\~\\}  % tex.stackexchange.com/questions/11622

%% Temas %%
% Tema y tema de color
\usetheme{Dresden}
\usecolortheme{beaver}


% Enlaces (tex.stackexchange.com/questions/13423)
\hypersetup{colorlinks,linkcolor=,urlcolor=links}


%% Título y otros %%
\title{Cómo usar \texttt{beamer}}  % Título
\subtitle{Una guía escrita en \texttt{beamer}}   % Subtítulo
\author[Pablo Baeyens]{Pablo Baeyens Fernández \\ \href{mailto:pbaeyens31+github@gmail.com}{pbaeyens31+github@gmail.com}} %Autor y e-mail
\date{DGIIM} % Fecha


%% Presentación %%
\begin{document}

% Página de título
\frame{\titlepage}

\frame{
  \frametitle{Índice}
  \tableofcontents[currentsection]
}

\section{Introducción}

\frame{
\frametitle{¡Contribuye!}
  El código fuente de éstas diapositivas está disponible en:
\begin{center}
  \huge \href{http://github.com/pbaeyens/beamer}{github.com/pbaeyens/beamer}
\end{center}
  Erratas, correcciones y aportaciones son bienvenidas.
}


\frame{
  \frametitle{¿Qué es \texttt{beamer}?}
  \texttt{beamer} es una clase de documento de $\LaTeX$ que genera diapositivas
  o transparencias con animaciones, gráficos, tablas...

  Puede compilarse con casi \footnote{\texttt{pandoc} no sirve \frownie{}} cualquier compilador
  de $\LaTeX$ y personalizarse al detalle.

}

\frame{
  \frametitle{Instalación}

  Para usar \texttt{beamer} se necesitan 3 paquetes:

  \begin{itemize}
    \item \texttt{beamer}
    \item \texttt{pgf}
    \item \texttt{xcolor}
  \end{itemize}

  En distros derivadas de Debian podemos instalar el paquete \texttt{latex-beamer} que
  gestionará las dependencias.

  También podemos utilizar \href{http://www.ctan.org/pkg/texliveonfly}{\texttt{texliveonfly}}.
}

\frame{
  \frametitle{Primeros pasos}

  Para empezar a usar \texttt{beamer}, indicamos la clase del documento:
    \espacio
    { \Large \centering \texttt
  {\color{black} \textbackslash}{\color{keywords}documentclass}{\color{black}\{beamer\}}
    }
    \espacio

  \texttt{beamer} utiliza una letra sin serifa por defecto para las fórmulas matemáticas.
  Podemos utilizar la fuente de la clase \texttt{article} pasando el argumento \texttt{mathserif}.
}

\frame[allowframebreaks]{
  \frametitle{Diapositivas}
  En el entorno \texttt{document} creamos las diapositivas utilizando el
  entorno \texttt{frame}:

  \ejemplo{ejemplo1.tex}

  También podemos utilizar \texttt{frame} como una función.
}

\muestra{ejemplo1.tex}


\frame{
  \frametitle{Tipos de diapositivas}
  El entorno \texttt{frame} permite varias opciones:
}

\muestra{shrink.tex}

\frame{
\frametitle{Tamaño}
\begin{columns}
  \column{.5\textwidth}
    Podemos cambiar el tamaño de letra utilizando los comandos habituales en $\LaTeX$.

  \column{.5\textwidth}
    \begin{itemize}
      \item \texttt{\tiny \textbackslash tiny}
      \item \texttt{\scriptsize \textbackslash scriptsize}
      \item \texttt{\footnotesize \textbackslash footnotesize}
      \item \texttt{\small \textbackslash small}
      \item \texttt{\normalsize \textbackslash normalsize}
      \item \texttt{\large \textbackslash large}
      \item \texttt{\Large \textbackslash Large}
      \item \texttt{\LARGE \textbackslash LARGE}
      \item \texttt{\huge \textbackslash huge}
    \end{itemize}
\end{columns}
}

% Secciones
\section{Gráficos y otras maravillas de \texttt{tikz}}

\frame{
\begin{columns}[c]
  \column{.6\textwidth}
    \begin{tikzpicture}
      \begin{axis}[
      ybar stacked,
      bar width=.7cm,
      ytick=\empty,
      symbolic x coords={\texttt{beamer},LibreOffice,PowerPoint},
      xtick=data,
      ]
        \addplot[fill=bars] coordinates
        {(\texttt{beamer},100) (LibreOffice,9) (PowerPoint,4)};
      \end{axis}
    \end{tikzpicture}

  \column{.4\textwidth}
   Podemos hacer gráficos con \href{https://ctan.org/pkg/pgfplots}{\texttt{pgfplots}} (aunque de este paquete se puede hablar tanto como de \texttt{beamer}).
  \end{columns}
}
\end{document}
